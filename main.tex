%----------------------------------------------------------------------------------------
%	PACKAGES AND OTHER DOCUMENT CONFIGURATIONS
%----------------------------------------------------------------------------------------

\documentclass[portrait,a0paper]{baposter} % Adjust the font scale/size here

\usepackage{minted} % Required for code blocks
\usepackage{amsmath} % Required for symbols

\usepackage{lipsum} % just for dummy text
\usepackage{url} % for url links

\definecolor{lightblue}{rgb}{0.145,0.6666,1} % Defines the color used for content box headers

%----------------------------------------------------------------------------------------
%	begin document
%----------------------------------------------------------------------------------------
\begin{document}

%----------------------------------------------------------------------------------------
%	poster style configuration
%----------------------------------------------------------------------------------------
\begin{poster}
{   
% keyval options
% border options %%%%%%%%%%%%%%%%%%%%%%%%%%%%%%%
headerborder=closed, % Adds a border around the header of content boxes
textborder=rounded, % Format of the border around content boxes, can be: none, bars, coils, triangles, rectangle, rounded, roundedsmall, roundedright or faded
linewidth=2pt, % Width of the border lines around content boxes
% layout options %%%%%%%%%%%%%%%%%%%%%%%%%%%%%%%
columns=3,
colspacing=1em, % Column spacing
% color %%%%%%%%%%%%%%%%%%%%%%%%%%%%%%%
bgColorOne=white, % Background color for the gradient on the left side of the poster
bgColorTwo=white, % Background color for the gradient on the right side of the poster
borderColor=lightblue, % Border color
headerColorOne=black, % Background color for the header in the content boxes (left side)
headerColorTwo=lightblue, % Background color for the header in the content boxes (right side)
headerFontColor=white, % Text color for the header text in the content boxes
boxColorOne=white, % Background color of the content boxes
% logo %%%%%%%%%%%%%%%%%%%%%%%%%%%%%%%
eyecatcher=false, % Set to false for ignoring the left logo in the title and move the title left
% block headers %%%%%%%%%%%%%%%%%%%%%%%%%
headerheight=0.1\textheight, % Height of the header
headershape=rounded, % Specify the rounded corner in the content box headers, can be: rectangle, small-rounded, roundedright, roundedleft or rounded
headerfont=\Large\bf\textsc, % Large, bold and sans serif font in the headers of content boxes
%textfont={\setlength{\parindent}{1.5em}}, % Uncomment for paragraph indentation
}
{   
% eyecatcher aka logo options, set eyecatcher=true if you need logos
}
{
% poster title
My Cheat Sheet
}
{
% poster subtitle
\textit{subheader here, incl name}
}
{
% university logo?
}
%----------------------------------------------------------------------------------------
%	poster contents
%----------------------------------------------------------------------------------------
\begin{posterbox}[name=box0,column=0,row=0]{Getting Started}
To install some library inside a virtual environment:
\begin{minted}{bash}
python -m pip install mylib
\end{minted}

To use some library in a script:
\begin{minted}{python}
import mylib as lib
\end{minted}
\end{posterbox}

%%%%%%%%%%%%%%%%%%%%%%%%%%%%%%%%%%%%%%%%%%%%%%%%%%%%
\begin{posterbox}[name=box1,below=box0,span=2]{A Table}

Here is a basic table template.

\resizebox{\textwidth}{!}{%
\begin{tabular}{|l|l|l|}
\hline
\textbf{Method} & \textbf{Notes} & \textbf{Output} \\ \hline
\textit{method1()} & I can't figure out how to put code blocks in a table. & int \\ \hline
\textit{method2(start:int=0)} & This is the best I can do for now. & float \\ \hline
\end{tabular}%
}
\end{posterbox}
%%%%%%%%%%%%%%%%%%%%%%%%%%%%%%%%%%%%%%%%%%%%%%%%%%%%
\begin{posterbox}[name=box2,span=1,align=box0,column=1]{BaPoster Documentation}
\url{www.brian-amberg.de/uni/poster/baposter/baposter_guide.pdf}
\end{posterbox}

%%%%%%%%%%%%%%%%%%%%%%%%%%%%%%%%%%%%%%%%%%%%%%%%%%%%
\begin{posterbox}[name=box3,span=1,aligned=box0,column=2]{Here's a lot of text!}
\lipsum[1-2]
\begin{minted}{python}
class Lipsum:
    def __init__(self, source):
        self.source = source
    
    def genr(self, k):
        k_words = self.source.split()
        k_words = k_words[:k]
        return " ".join(k_words)
\end{minted}
\lipsum[3]
\begin{minted}{python}
>>> lsm = Lipsum(LOREM_IPSUM)
>>> lsm.genr(5)
"Lorem ipsum dolor sit amet,"
\end{minted}
\end{posterbox}
\end{poster}
\end{document}
